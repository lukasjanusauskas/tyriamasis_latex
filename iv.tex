\paragraph{} Vaisingumo rodiklis apibūdinamas kaip vaikų skaičius, padalintas iš vaisingo amžiaus moterų skaičiaus. Jei bendras vaisingumo rodiklis yra 2,1 vaiko vienai moteriai, tai užtikrinama stabili populiacija(darant prielaidą, kad migracijos balansas nėra neigiamas ir mirtingumas nepakitęs). Taip vaisingumą apibūdina \hyperlink{oecd}{OECD}.
\hypertarget{sen_lt_vis}{\textcite{sensltvis}} analitinėje apžvalgoje minima, kad mažėjantis vaisingumo rodiklis yra vienas iš faktorių prisidedančių prie Lietuvos visuomenės senėjimo.
Todėl šiame tyrime aptariame sąryšius tarp įvairių kitų rodiklių ir vaisingumo rodiklio. \\
Tarp jų yra:
\begin{enumerate}
    \item \textbf{Žmogaus socialinės raidos indeksas} (ŽSRI) - tai apibendrinantis rodiklis, nusakantis pagrindines žmonių socialinės raidos dimensijas. Literatūroje dažnai teigiama, kad visuomenės socialinė raida neigiamai koreliuoja su vaisingumu (pavyzdžiui, Geetos \textcite{nargund2009declining} straipsnyje).
    \item \textbf{Mirtingumas}. Šis rodiklis apibrėžiamas kaip santykis mirčių ir asmenų tam tikroje populiacijoje per tam tikrą laikotarpį \hyperlink{merriam}{(apibūdinimas iš Merriam-Webster žodyno)}. 
    \item \textbf{Aukštojo išsilavinimo procentas}. Tai yra kaip procentas asmenų, kurių amžius yra tarp 25 ir 65 metų, įgijusių aukštesnįjį išsilavinimą.
\end{enumerate}