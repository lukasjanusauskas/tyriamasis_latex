\paragraph{} Vaisingumo rodiklis apibūdinamas kaip vaikų skaičius, padalintas iš vaisingo amžiaus moterų skaičiaus. Jei bendras vaisingumo rodiklis yra 2,1 vaiko vienai moteriai, tai užtikrinama stabili populiacija(darant prielaidą, kad migracijos balansas nėra neigiamas ir mirtingumas nepakitęs). Taip vaisingumą apibūdina \hyperlink{oecd}{OECD}.
\hypertarget{sen_lt_vis}{\textcite{sensltvis}} analitiniame tyrime minima, kad mažėjantis vaisingumo rodiklis yra vienas iš faktorių prisidednačių prie Lietuvos visuomenės senėjimo.
Todėl šiame tyrime aptariame sąryšius tarp įvairių kitų rodiklių ir vaisingumo rodiklio. \\
Tarp jų yra:
\begin{enumerate}
    \item Žmogaus socialinės raidos indeksas (ŽSRI) tai apibendrinantis rodiklis, nusakantis pagrindines žmonių socialinės raidos dimensijas. Literatūroje dažnai teigiama, kad visuomenės socialinė raida neigiamai koreliuoja su vaisingumu \textcite{nargund2009declining}.
    \item Mirtingumas apibrėžiamas kaip santykis mirčių ir asmenų tam tikroje populiacijoje per tam tikrą laikotarpį \hyperlink{merriam}{(apibūdinimas iš Merriam-Webster žodyno)}. 
    \item Aukštojo išsilavinimo procentas išreiškiamas kaip procentas asmenų, įgijusių aukščiausią išsilavinimą pagal amžiaus grupes.
\end{enumerate}