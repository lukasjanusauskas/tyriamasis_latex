\documentclass[12pt]{article}
\usepackage{graphicx}
\usepackage{amsmath}
\usepackage[utf8]{inputenc}
\usepackage[T1]{fontenc}
\usepackage[lithuanian]{babel}
\usepackage[margin=2cm]{geometry}
\usepackage{array}
\usepackage{multicol}
\usepackage{float}
\usepackage[backend=biber]{biblatex}
\usepackage{hyperref}
\hypersetup{
    colorlinks=true,
    linkcolor=blue,
    filecolor=magenta,      
    urlcolor=cyan,
    pdftitle={Overleaf Example},
    pdfpagemode=FullScreen,
    }

\addbibresource{lit.bib}

\newcommand{\diagrama}[2]{
\begin{figure}[H]
    \centering
    \includegraphics[width=.5\textwidth]{#1}
    \caption{#2}
\end{figure}
}

\begin{document}

\begin{titlepage}
    \centering
    \includegraphics[width=.5\textwidth]{pic/vu_logo.png} \\
    {\Huge
    Vilniaus Universitetas \\
    Matematikos ir Informatikos fakultetas\par} 
    \vspace{2cm}
    
    \Large
    {I grupė, I grupelė \\
    Arnas Usonis, Emilija Abromaitytė, \\ Igor Ziuganov, Lukas Janušauskas \\
    Recenzentai: VI grupelė\par}
    \vspace{1cm}
    {\LARGE
    \textbf{Tiriamasis projektas \\}}
    \vspace{0.2cm}
    {\Large
    Vaisingumo rodiklio ir susijusių rodiklių sąryšiai \\ bei pasiskirstymas pasaulyje. \\}
    \vspace*{\fill}
    Vilnius \\ 2023
\end{titlepage}
\pagebreak

\normalsize
\tableofcontents
\pagebreak

\section{Žymenys, santrumpos ir formulės}
\begin{itemize}
    \item ŽSRI - Žmogaus socialinės raidos indeksas(angl. Human development index)
    \item $r$ - Pirsono(Pearson) koreliacijos koeficientas.
    \item $\bar{x}$ - vidurkis.
\end{itemize}
\pagebreak

\section{Įvadas}
\paragraph{} Vaisingumo rodiklis apibūdinamas kaip vaikų skaičius, padalintas iš vaisingo amžiaus moterų skaičiaus. Jei bendras vaisingumo rodiklis yra 2,1 vaiko vienai moteriai, tai užtikrinama stabili populiacija(darant prielaidą, kad migracijos balansas nėra neigiamas ir mirtingumas nepakitęs). Taip vaisingumą apibūdina \hyperlink{oecd}{OECD}.
\hypertarget{sen_lt_vis}{\textcite{sensltvis}} analitinėje apžvalgoje minima, kad mažėjantis vaisingumo rodiklis yra vienas iš faktorių prisidedančių prie Lietuvos visuomenės senėjimo.
Todėl šiame tyrime aptariame sąryšius tarp įvairių kitų rodiklių ir vaisingumo rodiklio. \\
Tarp jų yra:
\begin{enumerate}
    \item \textbf{Žmogaus socialinės raidos indeksas} (ŽSRI) - tai apibendrinantis rodiklis, nusakantis pagrindines žmonių socialinės raidos dimensijas. Literatūroje dažnai teigiama, kad visuomenės socialinė raida neigiamai koreliuoja su vaisingumu (pavyzdžiui, Geetos \textcite{nargund2009declining} straipsnyje).
    \item \textbf{Mirtingumas}. Šis rodiklis apibrėžiamas kaip santykis mirčių ir asmenų tam tikroje populiacijoje per tam tikrą laikotarpį \hyperlink{merriam}{(apibūdinimas iš Merriam-Webster žodyno)}. 
    \item \textbf{Aukštojo išsilavinimo procentas}. Tai yra kaip procentas asmenų, kurių amžius yra tarp 25 ir 65 metų, įgijusių aukštesnįjį išsilavinimą.
\end{enumerate}
\pagebreak

\section{Tyrimo tikslas ir uždaviniai}
\subsection{Tikslas} 
Ištirti vaisingumo rodiklį ir susijusius rodiklius, bei ištirti šių kintamųjų sąryšius.

\subsection{Uždaviniai}
\begin{enumerate}
    \item Ištirti ŽSRI ir vaisingumo rodiklio išskirtis.
    \item Įvertinti ŽSRI ir vaisingumo rodiklio bei aušktojo išsilavinimo procento ir vaisingumo rodklio koreliacijas.
    \item Apibūdinti vaisingumo rodiklio skirstinį.
    \item Nustatyti vaisingumo pasiskirstymą šalyse ir žemynuose.
    \item Ištirti ar(ir kaip) skiriasi šie rodikliai Baltijos šalyse.
    \item Nustatyti, kaip keitėsi vaisingumas Baltijos šalyse tarp 1960 ir 2021 metų.
    \item Ištirti, koks yra sąryšis tarp mirtingumo ir vaisingumo, taip pat ištirsime, kaip šis sąryšis keitėsi, einant metams.
\end{enumerate}
\pagebreak

\section{Duomenys}
\textbf{Šiame tyrime naudojome 4 duomenų rinkinius iš 3 šaltinių}
\begin{itemize}
    \item Pasaulio banko. Duomenys apie pasaulio šalių ir regionų vaisingumo rodiklio duomenys nuo 1960 iki 2021 metų. Taip pat, panaudojome duomenis apie mirtingumo lygį įvairiose pasaulio šalyse nuo 1960 iki 2021 metų. 
    \item UN Socialinės Raidos Ataskaitos (angl. Human Development Reports) duomenys apie Žmonių Socialinės Raidos Indeksą(ŽSRI).
    \item Portalo ourworldindata.org duomenys, nusakantys kiekvienos šalies aukštojo išsilavinimo procentą nuo 1870 iki 2020, penkių metų intervalu. Taip pat, duomenų rinkinyje įtrauktos projekcijos iki 2040 metų.
\end{itemize}
\pagebreak

\section{Tyrimas}
\subsection{Išskirčių tyrimas}
\paragraph{} Tyrėme ŽSRI ir vaisinugmo rodiklio išskirtis, norėdami vėliau vykdyti sąryšių analizę, nes kai kurie duomenys gali iškraipyti kai kuriuos skaičiavimus(pavyzdžiui, Pirsono(angl. Pearson) koreliacijos koeficiento). 
Išskirtis tyrėme pasitelkdami kvartilių metodą. Tai reiškia, kad reikšmę laikėme sąlygine išskirtimi, jei ji priklausė intervalui $[Q1-3IQR; Q1-1,5IQR) \cup (Q3+1,5IQR; Q3+3IQR]$. O išskirtimi laikėme tokią reikšmę, kuri priklauso intervalui $(-\infty; Q1-3IQR] \cup [Q3+3IQR; \infty)$. \\\par
Šių kintamųjų išskirtis ir pasiskirstymą iliustruoja stačiakampės diagramos:

\begin{multicols}{2}
    \begin{figure}[H]
        \centering
        \includegraphics[width=.3\textwidth]{pic/box.png}
        \caption{ŽSRI stačiakampė diagrama}
    \end{figure}
    \begin{figure}[H]
        \centering
        \includegraphics[width=.3\textwidth]{pic/box2.png}
        \caption{Vaisingumo rodiklio stačiakampė diagrama}
    \end{figure}
\end{multicols}

Išskirčių lentelės:
\begin{multicols}{2}
\begin{table}[H]
\begin{center}
    \caption{ŽSRI santykinės išskirtys}
    \begin{tabular}{c c}
        \hline
        \textbf{Šalis} & \textbf{ŽSRI} \\\hline
        Šiaurės Korėja & 0 \\
        Somalija & 0 \\
        Nauru & 0 \\
        Monakas & 0 \\\hline
    \end{tabular}
\end{center}
\end{table}

\begin{table}[H]
\begin{center}
    \caption{Vaisingumo rodiklio santikinės išskirtys}
    \begin{tabular}{c c}
        \hline
        \textbf{Šalis} & \textbf{Vaisingumo rodiklis} \\\hline
        Nigeris & 6.892 \\\hline
    \end{tabular}
\end{center}
\end{table}
\end{multicols}

Vienintelės ŽSRI rodiklio išskirtys buvo šalys, kurių ŽSRI rodiklis buvo lygus 0, taigi jo tiesiog nebuvo. Todėl, vėliau atlikdami koreliacijos skaičiavimus, šiuos duomenis išfiltravom. Tuo tarpu, vaisingumo rodiklio vienintelė išskirtis buvo santykinė ir nebuvo tuščias duomuo, bet tiesiog gana didelė reikšmė, todėl jos iš skaičiavimų neišfiltravom.

\pagebreak


\subsection{Koreliacijos tyrimas}
\paragraph{} Siekdami nustatyti sąryšį tarp įvairių kintamųjų, pamatavome vaisingumo ir ŽSRI bei vaisingumo ir aukštojo išsilavinimo procento rodiklių koreliacijas. Tai padarėme, pritaikydami Pirsono(angl. Pearson) koreliacijos koeficientą. Šis koeficientas nusako \textbf{tiesinio} ryšio stiprumą ir kryptį.

Koreliacijos koeficientų reikšmės pateiktos lentelėje:
\begin{table}[H]
\begin{center}
    \caption{Koreliacijos skaičiavimų rezultatai}
    \begin{tabular}{|c|c|}
        \hline
        \textbf{Lyginti kintamieji} & \textbf{Pirsono koreliacijos koeficientas} \\\hline
        ŽSRI ir vaisingumo rodiklis & -0,819 \\\hline
        Aukštojo išsilavinimo procentas ir vaisingumo rodiklis & -0,399 \\\hline
    \end{tabular}
\end{center}
\end{table}

Šių kintamų sąryšius iliustruoja sklaidos diagramos:
\begin{multicols}{2}
\diagrama{pic/vias_hdi_rod.png}{Gyventojų vaisingumo rodiklio ir ŽSRI sklaidos diagrama}
\diagrama{pic/gyv_vais_kor.png}{Gyventojų vaisingumo rodiklio ir aukštojo išsilavinimo procento sklaidos diagrama}
\end{multicols}

\pagebreak

\subsection{Aukštąjį išsilavinimą gavusiųjų suaugiusiųjų pasiskirtsymo tyrimas} 

\paragraph{} Siekdami ištirti, koks yra aukštąjį išsilavinimą gavusių žmonių procento skirstinys 2020 metais, nubrėžėme histogramą. Histogramoje vizualiai galima pastebėti, ar skirstinys yra simetriškas, o jei ne, į kurią pusę pakrypęs.
\diagrama{pic/histo.png}{Aukštąjį išsilavinimą gavusių procentas.}
\pagebreak

\subsection{Vaisingumo pasiskirstymas pasaulyje}
\paragraph{} Tirdami vaisingumo rodiklio pasiskirstymą pasaulio šalyse nubrėžėme žemėlapį. Jame šalis, kurių vertės yra virš vidurkio, žymėjome spalva monochromatine oranžnei, o šalis, kurių vaisingumo rodiklis yra žemesnis nei vidurkis, žymime spalva monochromatine mėlynai.
\diagrama{pic/map.png}{Vaisingumo rodiklis įvairiose pasaulio šalyse.}

\paragraph{} Jau žemėlapyje galime pastebėti tendenciją, kad Afrikoje vaisingumas didžiausias. Todėl, taip pat verta įvertinti vaisingumo pasiskirstymą žemynuose. Šį pasiskirstymą iliustruoja skritulinė diagramą:
\hypertarget{pie}{\diagrama{pic/pie_2020.png}{Skritulinė diagrama, vaizduojanti vaisingumą įvairiuose žemynuose/regionuose}}
\pagebreak

\subsection{Vaisingumo rodiklis Baltijos šalyse}
\paragraph{} Tirta, ar yra skirtumas tarp Baltijos ŽSRI, šalių vaisingumo rodiklio ir aukštojo išsilavinimo procento. Kiekvienas rodiklis padalintas iš atitinkamos maksimalios reikšmės. Taip mes šiuos tris rodiklius patalpinom į tą pačią skalę. Šį palyginimą iliustruoja stulpelinė diagrama:
\diagrama{pic/balt.png}{Įvairūs rodikliai Baltijos šalyse, 2020 metais}

Taip pat, tirta, ar panašumas tarp Baltijos šalių vaisingumo rodiklių buvo ir seniau(1960-2021 metų intervale). Kadangi pokytis geriausiai matosi linijinėje diagramoje, pasirinktas būtent šis diagramos tipas. X ašyje pateikti metai, o Y ašyje - vaisingumo rodikliai.
\diagrama{pic/vais_balt.png}{Baltijos šalių vaisingumo rodiklis 1960-2021 metais}
\pagebreak

\subsection{Mirtingumo ir vaisingumo sąryšis 1960-2021 metais}
\paragraph{} Taip pat buvo ištirtas sąryšis tarp mirtingumo ir vaisingumo. Apskaičiuotas Pirsono koreliacijos koeficientas, atskleidė, kokio stiprumo ir krypties yra tiesinis sąryšis(ir ar būtent toks yra). O nubrėžta sklaidos diagrama, iliustravo sąryšį(šiuo atveju, jo nebuvimą).
\diagrama{pic/mirt_vais_kor.png}{Mirtingumas ir vaisingumas 2020 metais}

Iš pirmos dalies netikėto rezultato prieita išvada, kad svarbu patikrinti, kaip šis koeficientas kito per tarp 1960 iki 2021. Pamatuotas ir palygintas Pirsono(angl. Pearson) koreliacijos koeficientas šiame laiko tarpe. Rezultatus pavaizduosime linijine diagrama:
\diagrama{pic/mirt_vais.png}{Koreliacijų koeficientai 1960-2021 metais}

Atlikę skaičiavimus pastebėjome, kad nuo 1960 iki 2000(paskutinių metų, kada Pirsono koreliacijos koeficientas buvo didesnis už 0.5) buvo teigiamas tiesinis sąryšis, tačiau nuo maždaug 2000-ųjų koreliacijos koeficientas pradėjo smukti. 
\pagebreak

\section{Išvados}
\begin{enumerate}
    \item \textbf{ŽSRI ir vaisingumo rodiklio išskirtys}
    Kadangi vaisingumo rodiklio išskirtys buvo tik sąlyginės ir atsirado dėl to, kad duomenys nebuvo . Galima sakyti, kad tikėtina, 

    \item \textbf{Vaisingumo rodiklis ir ŽSRI bei aukštojo išsilavinimo sąryšis}
    Galime daryti išvadą, kad vaisingumo rodiklis stipriai neigiamai koreliuoja su ŽSRI, tai reiškia, kad yra neigiamas tiesinis sąryšis tarp vaisingumo rodiklio ir ŽSRI. Todėl galima teigti, kad didėjant ŽSRI, krinta vaisingumo lygis.
    Tačiau vaisingumas su aukštojo išsilavinio procentu koreliuoja gana silpnai, nes koreliacijos koeficientas lygus -0.39. Tai reiškia, kad tiesinis sąryšis tarp aukštojo išsilavinimo procento ir vaisingumo yra neigiamas ir silpnas.

    \item \textbf{Vaisingumo rodiklio pasiskirtsymas}
    Iš histogramos matome, kad vaisingumo rodiklio pasiskirstymas, nėra simetriškas. Didžioji dalis šalių turi gana mažą vaisingumo rodiklį. 

    \item \textbf{Vaisingumas Baltijos šalyse}
    Baltijos šalyse minėti rodikliai yra labai panašūs. Taip pat, galima pastebėti, kad Baltijos šalių vaisingumas visais metais nuo 1960 iki 2021 buvo panašus ir kitimo tendencijos buvo praktiškai identiškos.

    \item \textbf{Vaisingumo pasiskirstymas pasaulyje}
    Iš žemėlapio matome, kad daugelis šalių, kurių vaisingumas yra didesnis už vidurkį, yra Afrikoje.
    Taip pat tirdami pagal žemynus, matome iš skritulinės daigramos (\hyperlink{pie}{žr. 7 pav.}), kad didžiausias vaisingumas yra Afrikos regionuose(išskyrus šiaurinį), o kitose yra gana panašus ir su minėtais regionais palyginus mažas.

    \item \textbf{Mirtingumo ir vaisingumo sąryšis 1960-2021 metais}
    Kadangi Pirsono koreliacijos koeficientas 2020 metais yra neigimas ir arti nulio, galima teigti, kad tiesinio sąryšio nėra, taip pat iš sklaidos grafiko panašu, kad jokio sąryšio(ir netiesinio) praktiškai nėra. Taip nebuvo visada: tarp 1960 ir 2000-ųjų pirmojo dešimtmečio pradžios tarp kintamųjų buvo teigiamas tiesinis sąryšis.
\end{enumerate}
\pagebreak

\section{Šaltiniai}
\begin{enumerate}
    \item Jungtinių Tautų ŽSRI duomenys \\
    \href{https://hdr.undp.org/data-center/human-development-index\#/indicies/HDI}{https://hdr.undp.org/data-center/human-development-index\#/indicies/HDI}
    \item \hypertarget{merriam}{Merriam-Webster žodynas} \\
    \href{https://www.merriam-webster.com/dictionary/mortality\%20rate}{https://www.merriam-webster.com/dictionary/mortality\%20rate}
    \item \hypertarget{oecd}{OECD „Fertility rates“} \\
    \href{https://data.oecd.org/pop/fertility-rates.htm}{https://data.oecd.org/pop/fertility-rates.htm}
    \item Pasaulio Bankas „Fertility rate, total (births per woman)“ \\
    \href{https://data.worldbank.org/indicator/SP.DYN.TFRT.IN}{https://data.worldbank.org/indicator/SP.DYN.TFRT.IN}
    \item Portalo \url{ourworldindata.org} aukštesniojo išsilavinimo procento duomenys. \\
    \href{https://ourworldindata.org/grapher/share-of-the-population-with-completed-tertiary-education?tab=table}{https://ourworldindata.org/grapher/share-of-the-population-with-completed-tertiary-education?tab=table}
\end{enumerate}


\printbibliography

\end{document}
