\begin{enumerate}
    \item \textbf{ŽSRI ir vaisingumo rodiklio išskirtys}
    Kadangi vaisingumo rodiklio išskirtys buvo tik sąlyginės ir atsirado dėl to, kad duomenys nebuvo . Galima sakyti, kad tikėtina, 

    \item \textbf{Vaisingumo rodiklis ir ŽSRI bei aukštojo išsilavinimo sąryšis}
    Galime daryti išvadą, kad vaisingumo rodiklis stipriai neigiamai koreliuoja su ŽSRI, tai reiškia, kad yra neigiamas tiesinis sąryšis tarp vaisingumo rodiklio ir ŽSRI. Todėl galima teigti, kad didėjant ŽSRI, krinta vaisingumo lygis.
    Tačiau vaisingumas su aukštojo išsilavinio procentu taip stirpiai nekoreliuoja, kadangi koreliacijos koeficientas lygus -0.39. Tai reiškia, kad tiesinis sąryšis tarp aukštojo išsilavinimo procento ir vaisingumo yra neigiamas ir silpnas.

    \item \textbf{Vaisingumo rodiklio pasiskirtsymas}
    Iš histogramos matome, kad vaisingumo rodiklio pasiskirstymas, nėra simetriškas. Didžioji dalis šalių turi gana mažą vaisingumo rodiklį. 

    \item \textbf{Vaisingumas Baltijos šalyse}
    Baltijos šalyse minėti rodikiai yra labai panašūs. Taip pat, galima pastebėti, kad Baltijos šalių vaisingumas visais metais nuo 1960 iki 2021 buvo panašus ir kitimas buvo praktiškai identiškas.

    \item \textbf{Vaisingumo pasiskirstymas pasaulyje}
    Iš žemėlapio matome, kad daugelis šalių, kurių vaisingumas yra didesnis už vidurkį, yra Afrikoje.
    Taip pat tirdami pagal žemynus, matome iš skirtulinio grafiko, kad didžiausias vaisingumas yra Afrikoje.

    \item \textbf{Mirtingumo ir vaisingumo sąryšis 1960-2021 metais}
    Kadangi Pirsono koreliacijos koeficientas 2020 metais yra neigimas ir arti nulio, galima teigti, kad tiesinio sąryšio nėra, taip pat iš sklaidos grafiko panašu, kad jokio sąryšio(ir netiesinio) praktiškai nėra. Taip nebuvo visada: tarp 1960 ir 2000-ųjų pirmojo dešimtmečio pradžios tarp kintamųjų buvo teigiamas teisinis sąryšis.
\end{enumerate}