\begin{enumerate}
    \item \textbf{ŽSRI ir vaisingumo rodiklio išskirtys ir skirstiniai}
    Kadangi išskirtys buvo tik sąlyginės ir kai kurios buvo dėl duomenų trūkumo. Skirstiniai buvo akivaizdžiai nesimetriški, todėl galima sakyti, kad daugelio šalių vaisingumas yra palyginus mažas, o didelis vaisingumas šalyje yra retas. 
    \end{itemize}

    \item \textbf{Vaisingumo rodiklis ir ŽSRI bei aukštojo išsilavinimo sąryšis}
    Galime daryti išvadą, kad vaisingumo rodiklis stipriai neigiamai koreliuoja su ŽSRI, tai reiškia, kad yra neigiamas tiesinis sąryšis tarp vaisingumo rodiklio ir ŽSRI. \\
    Tačiau vaisingumas su aukštojo išsilavinio procentu taip stirpiai nekoreliuoja, kadangi koreliacijos koeficientas lygus -0.39

    \item \textbf{Vaisingumo rodiklio pasiskirtsymas}
    Iš histogramos matome, kad vaisingumo rodiklio pasiskirstymas, nėra simetriškas. Didžioji dalis šalių turi gana mažą vaisingumo rodiklį. 

    \item \textbf{Vaisingumas Baltijos šalyse}
    Baltijos šalyse minėti rodikiai yra labai panašūs. Taip pat, galima pastebėti, kad Baltijos šalių vaisingumas visais metais nuo 1960 iki 2021 buvo panašus ir kitimas buvo praktiškai identiškas.

    \item \textbf{Vaisingumo pasiskirstymas pasaulyje}
    Iš žemėlapio matome, kad daugelis šalių, kurių vaisingumas yra didesnis už vidurkį, yra Afrikoje.
    Taip pat tirdami pagal žemynus, matome iš skirtulinio grafiko, kad didžiausias vaisingumas yra Afrikoje.
\end{enumerate}