\begin{enumerate}
    \item \textbf{HDI ir vaisingumo rodiklio išskirtys}
        \begin{itemize}
        \item \textbf{HDI rodiklio išskirtys} - vienintėlės HDi rodiklio išskirtys buvo šalys, kurių HDI rodiklis nėra duotas. Todėl  
        \item \textbf{Vaisingumo rodiklio išskirtys} - 
    \end{itemize}

    \item \textbf{Vaisingumo rodiklis ir HDI bei aukštojo išsilavinimo sąryšis}
    Galime daryti išvadą, kad vaisingumo rodiklis stipriai neigiamai koreliuoja su HDI, tai reiškia, kad yra neigiamas tiesinis sąryšis tarp vaisingumo rodiklio ir HDI. \\
    Tačiau vaisingumas su aukštojo išsilavinio procentu taip stirpiai nekoreliuoja, kadangi koreliacijos koeficientas lygus -0.39

    \item \textbf{Vaisingumo rodiklio pasiskirtsymas}
    Iš histogramos matome, kad vaisingumo rodiklio pasiskirstymas, nėra simetriškas. Didžioji dalis šalių turi gana mažą vaisingumo rodiklį. 

    \item \textbf{Vaisingumas Baltijos šalyse}
    Baltijos šalyse minėti rodikiai yra labai panašūs. Taip pat, galima pastebėti, kad Baltijos šalių vaisingumas visais metais nuo 1960 iki 2021 buvo panašus ir kitimas buvo praktiškai identiškas.

    \item \textbf{Vaisingumo pasiskirstymas pasaulyje}
    Iš žemėlapio matome, kad daugelis šalių, kurių vaisingumas yra didesnis už vidurkį, yra Afrikoje.
    Taip pat tirdami pagal žemynus, matome iš skirtulinio grafiko, kad didžiausias vaisingumas yra Afrikoje.
\end{enumerate}