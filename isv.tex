\begin{enumerate}
    \item \textbf{ŽSRI ir vaisingumo rodiklio išskirtys}
    \begin{itemize}
        \item ŽSRI išskirtys buvo tik tų šalių, kurių ŽSRI lygus nuliui. Iš to galima nuspėti, kad duomenų šiose šalyse surinkti negalėjo ir todėl šie duomenys tyrimui nėra svarbūs. Daugiau svarbių išskirčių nebuvo. 
        \item Vaisingumo rodiklio išskirtis buvo viena - Nigerio labai aukštas vaisingumo roodiklis. Iš to galime daryti išvadą, kad Nigerio gyventojų vaisingumo rodiklis yra išskirtinai didelis.
    \end{itemize}

    \item \textbf{Vaisingumo rodiklis ir ŽSRI bei aukštojo išsilavinimo sąryšis}
    Galime daryti išvadą, kad vaisingumo rodiklis stipriai neigiamai koreliuoja su ŽSRI, tai reiškia, kad yra neigiamas tiesinis sąryšis tarp vaisingumo rodiklio ir ŽSRI. Todėl galima teigti, kad didėjant ŽSRI, krinta vaisingumo lygis.
    Tačiau vaisingumas su aukštojo išsilavinimo procentu koreliuoja gana silpnai, nes koreliacijos koeficientas lygus -0.39. Tai reiškia, kad tiesinis sąryšis tarp aukštojo išsilavinimo procento ir vaisingumo yra neigiamas ir silpnas.

    \item \textbf{Vaisingumo rodiklio pasiskirtsymas}
    Iš histogramos matome, kad vaisingumo rodiklio pasiskirstymas, nėra simetriškas. Didžioji dalis šalių turi gana mažą vaisingumo rodiklį. 

    \item \textbf{Vaisingumas Baltijos šalyse}
    Baltijos šalyse minėti rodikliai yra labai panašūs. Taip pat, galima pastebėti, kad Baltijos šalių vaisingumas visais metais nuo 1960 iki 2021 buvo panašus ir kitimo tendencijos buvo praktiškai identiškos.

    \item \textbf{Vaisingumo pasiskirstymas pasaulyje}
    Iš žemėlapio matome, kad daugelis šalių, kurių vaisingumas yra didesnis už vidurkį, yra Afrikoje.
    Taip pat tirdami pagal žemynus, matome iš skritulinės daigramos (\hyperlink{pie}{žr. 7 pav.}), kad didžiausias vaisingumas yra Afrikos regionuose(išskyrus šiaurinį), o kitose yra gana panašus ir su minėtais regionais palyginus mažas.

    \item \textbf{Mirtingumo ir vaisingumo sąryšis 1960-2021 metais}
    Kadangi Pirsono koreliacijos koeficientas 2020 metais yra neigimas ir arti nulio, galima teigti, kad tiesinio sąryšio nėra, taip pat iš sklaidos grafiko panašu, kad jokio sąryšio(ir netiesinio) praktiškai nėra. Taip nebuvo visada: tarp 1960 ir 2000-ųjų pirmojo dešimtmečio pradžios tarp kintamųjų buvo teigiamas tiesinis sąryšis.
\end{enumerate}