\subsection{Išskirčių tyrimas}
Tyrėme HDI ir vaisinugmo rodiklio išskirtis.
Išskirtis tyrėme pasitelkdami kvartilių metodą. Tai reiškia, kad reikšmę laikėme sąlygine išskirtimi, jei ji prikluasė intervalui $[Q1-3IQR; Q1-1,5IQR) \cup (Q3+1,5IQR; Q3+3IQR]$.

\subsection{Koreliacijos tyrimas}
Pamatavome vaisingumo ir HDI bei vaisingumo ir aukštojo išsilavinimo procento rodiklių koreliacijas. Tai padarėme, pritaikydami Pirsono(angl. Pearson) koreliacijos koeficientą. \\

Šių kintamų jų koreliacijas iliustruoja sklaidos diagramos:

\subsection{Aukštąjį išsilavinimą gavusiųjų suaugiusiųjų pasiskirtsymo tyrimas} 